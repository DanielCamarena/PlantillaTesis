\subsection*{Respuesta Sísmica de una Edificación con AS}
%\phantomsubsection
\addcontentsline{toc}{subsection}{Respuesta Sísmica de una Edificación con AS}

\begin{MyFont}
\begin{adjustwidth}{4.8mm}{}
\begin{lstlisting}[language=Python, caption= {\footnotesize Edificación con aislamiento sísmico bilineal}, mathescape=true,label={Algoritmo1}]
# REPUESTA SÍSMICA DE UNA EDIFICACIÓN DE n NIVELES CON AS
import Funciones_KS as fun
import numpy as np
from scipy  import linalg as LA
import copy

# PROPIEDADES DINÁMICAS
# Parámetros dinámicos de la Superestructura
n=3                 # N° de Pisos
gdl=n+1             # GDL = N° de Pisos+1
Tnf=0.3             # Periodo base fija, [s]
$\xi$=5           			 # Amortiguamiento, [%]
$\lambda$=(gdl-1)/gdl 			 # Relación de masas

# Parámetros dinámicos de la interfaz de aislamiento
r=0.1               # Razón de rigideces Kp/Ke
Q=105          		  # Fuerza característica normalizada, [cm/s^2]
K2=30         		  # Rigidez postfluencia normalizada, [1/s^2]

# LECTURA DEL REGISTRO SÍSMICO
ug=np.genfromtxt("./Sismo_Lima66NS.txt")  #[cm/s^2]
$\Delta$t=0.002; N=len(ug)
t=[i*$\Delta$t for i in range(N)]

\end{lstlisting}
\end{adjustwidth}
\end{MyFont}
